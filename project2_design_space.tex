\documentclass[11pt]{article}

\usepackage[margin=1in]{geometry}
\usepackage{graphicx}
\usepackage{tabularx}
\usepackage{amsmath}
\usepackage{booktabs}
\usepackage[hidelinks]{hyperref}

\title{A Design Space of Uncertainty and Explanation Views for Clinical Risk Dashboards}
\author{Gaurav}
\date{}

\begin{document}
\maketitle

\begin{abstract}
Clinical decision-support systems increasingly incorporate machine learning models to estimate patient risk. 
However, raw risk scores are difficult to interpret without corresponding uncertainty information or supporting explanations. 
We present a five-prototype design space illustrating how uncertainty and explanations can be visualized, combined, 
and composed into clinical dashboards. 
Across seven dimensions---task purpose, uncertainty type, visual encoding, cognitive complexity, granularity, 
explanation integration, and intended audience---we characterize how different representations support different clinical needs. 
The prototypes include distributional dotplots, hybrid interval--explanation views, risk spectrum bars, 
dashboard layouts, and small-multiple comparison views, forming a compact foundation for future exploration 
of uncertainty-aware clinical AI interfaces.
\end{abstract}

\section{Introduction}
Clinical AI systems increasingly provide risk scores for diagnosis and triage, yet these scores alone are insufficient 
without uncertainty and explainability. Uncertainty helps clinicians gauge the confidence of the estimate, while explanations 
help them understand why. Prior research in uncertainty visualization and explainable AI suggests multiple ways 
to encode uncertainty (intervals, distributions, stability) and explanations (feature contributions, SHAP values), 
but relatively little work integrates them into coherent design variations suitable for clinical dashboards.

This work presents a structured design space and five corresponding prototypes for visualizing uncertainty and explanations 
in clinical decision support. The design space offers a conceptual scaffold that practitioners and researchers can use 
to reason about how different combinations of encodings affect interpretability, cognitive load, and workflow fit.

We instantiate the design space using a heart-disease risk prediction task with three archetype patients: a confident low-risk case (LOW\_tight), 
an uncertain borderline case (MID\_wide), and a confident high-risk case (HIGH\_tight). 
All prototypes are implemented in Python and generate static figures meant to stand in for candidate dashboard views.

\section{Prototype Gallery}

\subsection{P3: Risk Spectrum Bar (Low-Complexity Triage View)}
Prototype 3 provides a low-complexity ``triage'' view: a color-coded low/medium/high risk bar 
overlaid with a 95\% confidence interval (CI) band and a point estimate. 
The clinician can quickly see whether the current patient is in a low, medium, or high risk region and whether the estimate is precise or uncertain.

\begin{figure}[htbp]
    \centering
    \includegraphics[width=0.75\linewidth]{figs/proto3_HEART_MID_wide.png}
    \caption{Prototype P3: risk spectrum bar for the MID\_wide heart-disease patient. 
    The background indicates low/medium/high risk zones, the gray band encodes the CI, 
    and the vertical line marks the mean risk.}
    \label{fig:p3}
\end{figure}

\subsection{P2: Interval + SHAP Hybrid}
Prototype 2 combines a simple risk interval (mean + CI) with a horizontal SHAP bar chart. 
The left panel shows an interval plot of predicted risk, while the right panel visualizes the top features 
sorted by absolute SHAP value. This supports quick understanding of both risk magnitude and its main drivers.

\begin{figure}[htbp]
    \centering
    \includegraphics[width=0.85\linewidth]{figs/proto2_HEART_MID_wide.png}
    \caption{Prototype P2: interval + SHAP hybrid for the MID\_wide patient. 
    The left interval plot shows the CI around the mean risk; the right panel shows feature contributions (SHAP values).}
    \label{fig:p2}
\end{figure}

\subsection{P1: Dotplot + Explanation Stability}
Prototype 1 visualizes prediction \emph{distribution} via quantile-style dotplots 
and explanation \emph{stability} via bootstrap-sampled SHAP values for the top feature. 
The left panel shows the distribution of predicted risk under bootstrap resampling of the model, 
while the right panel shows how the SHAP value of the most important feature varies across bootstraps.
This prototype targets detailed uncertainty inspection for data-savvy clinicians or researchers.

\begin{figure}[htbp]
    \centering
    \includegraphics[width=0.85\linewidth]{figs/proto1_HEART_MID_wide.png}
    \caption{Prototype P1: prediction and explanation uncertainty for the MID\_wide patient. 
    Left: dotplot of bootstrap predictions with CI markers. 
    Right: dotplot of bootstrap SHAP values for the top feature, visualizing explanation stability.}
    \label{fig:p1}
\end{figure}

\subsection{P4: Dashboard Layout (Integrated Card View)}
Prototype 4 integrates the previous elements into a single dashboard card: 
a risk spectrum bar at the top (summary), a prediction uncertainty dotplot in the middle (distribution), 
and SHAP feature contributions at the bottom (explanations). 
This prototype illustrates multi-level uncertainty communication within a single patient view.

\begin{figure}[htbp]
    \centering
    \includegraphics[width=0.75\linewidth]{figs/proto4_HEART_dashboard_MID_wide.png}
    \caption{Prototype P4: integrated dashboard layout for the MID\_wide patient. 
    The card combines a spectrum-based summary, a distributional uncertainty view, and feature-level explanations.}
    \label{fig:p4}
\end{figure}

\subsection{P5: Comparison View (Small Multiples Across Archetypes)}
Prototype 5 shows small-multiple uncertainty and explanation views across the three archetype patients 
(Confident low, Uncertain mid, Confident high). 
The top row uses compact CI plots; the bottom row shows tiny SHAP bar charts of the top features for each patient. 
This supports comparative assessment and teaching, helping clinicians calibrate their understanding of model behavior 
across qualitatively different cases.

\begin{figure}[htbp]
    \centering
    \includegraphics[width=0.95\linewidth]{figs/proto5_HEART_comparison_view.png}
    \caption{Prototype P5: comparative view across archetype patients (Confident low, Uncertain mid, Confident high). 
    Each column shows a compact interval plot and a mini SHAP bar chart, with semantic labels underneath.}
    \label{fig:p5}
\end{figure}

\section{Design Space Dimensions}
We organize the prototypes along seven design dimensions characterizing uncertainty and explanation visualization choices.

\begin{enumerate}
    \item \textbf{Primary Task / Purpose} -- triage, uncertainty inspection, explanation inspection, multi-level integration, comparison.
    \item \textbf{Uncertainty Type} -- prediction range (CI); prediction distribution (samples); explanation uncertainty (SHAP variability).
    \item \textbf{Visual Encoding} -- CI bars, CI bands, dotplots, color-coded risk zones, SHAP bars, small multiples.
    \item \textbf{Cognitive Complexity} -- low, medium, or high depending on clinician workload and time pressure.
    \item \textbf{Granularity} -- single-patient vs.\ multi-patient views.
    \item \textbf{Explainability Integration} -- none; hybrid risk + explanation; stability views; embedded dashboard layouts; comparative explanations.
    \item \textbf{Intended Audience} -- busy clinicians, trainees, or data-savvy clinicians and AI researchers.
\end{enumerate}

These dimensions emerged through iterative sketching and implementation of the five prototypes, as well as reflection on 
clinical scenarios observed in the literature.

\section{Design Space Summary Table}
Table~\ref{tab:designspace} summarizes how the five prototypes map onto the design dimensions.

% \begin{table}[htbp]
% \centering
% \small
% \begin{tabular}{p{0.8cm} p{3.2cm} p{3.0cm} p{3.6cm} p{1.5cm} p{1.4cm} p{3.2cm}}
% \toprule
% \textbf{Proto} & \textbf{Primary Purpose} & \textbf{Uncertainty Type} & \textbf{Encoding} & \textbf{Complexity} & \textbf{Granularity} & \textbf{Explainability Integration} \\
% \midrule
% P1 & Inspect uncertainty in prediction and top explanation & Distribution + SHAP variability & Dotplots (risk + SHAP) with CI markers & High & Single & Stability of top feature \\
% P2 & See risk range and \emph{why} in one glance & CI range & Interval line + SHAP bars & Medium & Single & Parallel hybrid (risk + SHAP) \\
% P3 & Quick triage under time pressure & CI range & Risk spectrum (color) + CI band & Low & Single & None (pure uncertainty) \\
% P4 & Integrated dashboard card & CI + distribution & Spectrum bar + dotplot + SHAP bars & Med--High & Single & Embedded explanations and uncertainty \\
% P5 & Compare archetype patients & CI across patients & Small-multiple CI plots + SHAP bars & Medium & Multi & Parallel per-patient explanations \\
% \bottomrule
% \end{tabular}
% \caption{Design space summary: five prototypes positioned along task purpose, uncertainty type, visual encoding, complexity, granularity, and explainability integration.}
% \label{tab:designspace}
% \end{table}

\begin{table}[htbp]
\centering
\small
\setlength{\tabcolsep}{3pt} % tighter horizontal padding
\begin{tabularx}{\textwidth}{l
                                    X
                                    X
                                    X
                                    l
                                    l
                                    X}
\toprule
\textbf{Proto} & \textbf{Primary Purpose} & \textbf{Uncertainty Type} &
\textbf{Encoding} & \textbf{Complexity} & \textbf{Granularity} &
\textbf{Explainability Integration} \\
\midrule
P1 & Inspect uncertainty in prediction and top explanation &
Distribution + SHAP variability &
Dotplots (risk and SHAP), CI markers &
High & Single & Stability of top feature \\
P2 & See range and why &
CI range &
Interval line + SHAP bars &
Medium & Single & Parallel hybrid (risk + SHAP) \\
P3 & Quick triage &
CI range &
Risk spectrum (color) + CI band &
Low & Single & None (uncertainty only) \\
P4 & Integrated dashboard &
CI + distribution &
Spectrum bar + dotplot + SHAP bars &
Med--High & Single & Embedded explanations in dashboard card \\
P5 & Compare archetypes &
CI across patients &
Small-multiple CI + small-multiple SHAP bars &
Medium & Multi & Parallel per-patient explanations \\
\bottomrule
\end{tabularx}
\caption{Summary of the five prototypes across key design dimensions.}
\label{tab:designspace}
\end{table}
    

\section{Discussion}
The five prototypes collectively demonstrate how uncertainty and explanation can be layered, separated, or combined 
depending on clinical information needs. Simpler views (P3) support rapid triage and could be embedded into existing electronic health record interfaces. 
Distributional and stability views (P1) support deeper sensemaking by data-savvy clinicians or AI teams who must audit model behavior. 
Composite designs (P4) illustrate how multi-level uncertainty and explanations can coexist in a single patient card, 
while comparative views (P5) reveal archetype-level patterns that clinicians might use for calibration or teaching.

The design space also highlights tensions. For example, adding explanation stability (P1) increases cognitive complexity, 
which may not be appropriate for time-pressured settings. Conversely, very simple encodings (P3) can hide important nuances, 
such as multimodal predictive distributions. Prototype 4 balances these tensions by offering both summary and detail views.

\section{Limitations and Future Work}
The current prototypes are conceptual and use simulated prediction and SHAP samples based on three representative heart-disease cases. 
We do not claim that any particular design is optimal. A full evaluation with clinicians would be needed to validate interpretability, 
trust calibration, and workflow fit. Future work could extend the design space to interactive views, integrate temporal trends, 
and explore patient-facing versions of uncertainty explanations.

\section{Conclusion}
This work proposes a five-prototype design space for uncertainty and explanation views in clinical risk dashboards. 
By articulating how uncertainty and explanation combine across visual encodings, complexity levels, and clinical tasks, 
the design space provides a structured starting point for future uncertainty-aware clinical AI interfaces. 
We hope the prototypes and dimensions will help both researchers and practitioners reason more systematically 
about how to communicate uncertainty and explanations in high-stakes decision support.

\end{document}
